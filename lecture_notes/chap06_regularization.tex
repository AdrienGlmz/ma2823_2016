\documentclass[a4paper,12pt]{article}

% Font
\usepackage[T1]{fontenc}
\usepackage{gentium}

% Math packages
\usepackage{amsmath}
\usepackage{amsfonts}
\usepackage{amssymb}
\usepackage{amsthm}
\usepackage{bm}

% Define symbol shortcuts
\newcommand{\cc}{\mathcal{C}}
\newcommand{\dd}{\mathcal{D}}
\newcommand{\hh}{\mathcal{H}}
\newcommand{\xx}{{\bm x}}
\newcommand{\yy}{{\bm y}}

% Math environment
\newtheorem*{thm}{Theorem}

% Better list management:
% - vertical spacing in lists
% - items in lists start with dash not bullet point.
\usepackage{enumitem}
\setlist{label=\textemdash,
  itemsep=0pt, topsep=3pt, partopsep=0pt} 

% Include graphics
\usepackage{graphicx}
\usepackage{subcaption}

% Page format 
\usepackage[top=2cm,left=2cm,right=2cm,bottom=2cm]{geometry}

\begin{document}
%%% HEADER
\raisebox{0.6in}[0in]{\makebox[\textwidth][r]{\it Unproofed version }}
\vspace{-0.7in}

\begin{center}
\bf\large MA2823: Foundations of Machine Learning \\
Chapter 6: Regularized Linear Regression
\end{center}

\noindent
Lecturer: Chlo\'e-Agathe Azencott   
\hfill
Scribe: Adrien Galamez \\
\null \hfill Paul Magon de la Villehuchet


\noindent
\rule{\textwidth}{1pt}

\medskip

In this chapter we will see:
\begin{enumerate}
\item what is regularization;
\item how to use regularization as a means to control model complexity;
\item several forms of regularizing linear regression models.
\end{enumerate}

\end{document}
